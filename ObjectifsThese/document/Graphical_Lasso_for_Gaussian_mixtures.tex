\section{Graphical Lasso for Gaussian mixtures}

The EM algorithm experiences severe performance degradation in high-dimensional setting. A technique to avoid this degradation is by regularizing the parameters of the model. The following algorithm is inspired by the Graphical lasso \citep{glasso07} \citep{banerjee} which penalizes the components of the precision matrix of a Gaussian graphical model.\

We consider $\bx_1,\bx_2,\dots,\bx_n$ a sample of $n$ points drawn from a $p$-dimensional Gaussian mixture distribution. In this problem, we will consider the estimation of $\btheta=(\theta_1,\dots,\theta_K)$ with $\theta_k=(\pi_k, \bmu_k, \bOmega_k)$ where $\bOmega_k$ is the precision matrix regarding the $k$ component of the mixture. We denote $\varphi_{(\bmu_{k},\bOmega_{k})}$ the gaussian density of mean $\bmu_k$ and precision matrix $\bOmega_k$. The penalized log-likelihood is
\begin{equation}\label{pen-log-likelihood}
\ell_n^{pen}(\btheta)=\sum_{i=1}^{n}\log p_{\btheta}(\bx_i)-pen(\btheta)= \sum_{i=1}^{n}\log \bigg\{ \sum_{k=1}^K\pi_k\varphi_{(\bmu_{k},\bOmega_{k})}(\bx_i)\bigg\} -pen(\btheta).
\end{equation}
In this problem, we suppose that within a cluster $k$, most pairs of features $(X^i,X^j)$ are independent given the other features $X^l \text{ with } i,j,l\in[p], l\notin\{i,j \}$. This property entails the sparsity of $\bOmega_k$. 
Therefore, we consider an $l_1$ regularization $pen(\theta_k)=\lambda_k||\bOmega_k||_{1,1}$ with $\lambda_k>0$.\

The penalization of the log-likelihood concerns only the precision matrices $\bOmega_k$. Regarding the other parameters $(\pi_k, \bmu_k)$, our algorithm is the same as EM and we can use the same iteration technique as in lemma \ref{lemma1} to maximize the following cost function
\begin{equation}
\label{cost_fun_pen}
F^{pen}(\btheta,\bTau)  = \sum_{k=1}^K \bigg(\sum_{i=1}^{n} \Big\{\tau_{i,k}\log\varphi_{\bmu_{k},\bOmega_{k}}(\bx_i)+\tau_{i,k}
    \log(\pi_k/\tau_{i,k})\Big\}-\lambda_k||\bOmega_k||_{1,1}\bigg)
\end{equation}

Then, we have the two following optimization problems
\begin{align}
\label{optim-problems}
\hat\btheta(\bTau)&\in \arg\max_{\btheta} F^{pen}(\btheta,\bTau),\qquad \hat\bTau(\btheta)\in \arg\max_{\bTau} F^{pen}(\btheta,\bTau)
\end{align}
which has explicit solutions. For a $\hat\bTau$ given, estimates of $(\pi,\bmu)$ obtained by the first optimization problem in \ref{optim-problems} are
\begin{align}
\label{em-sols}
\hat\pi_k     &= \frac1n\sum_{i=1}^n \hat\tau_{i,k},\qquad\hat\bmu_k = \frac1{n\hat\pi_k}\sum_{i=1}^n \hat\tau_{i,k}\bx_i ,\qquad \forall k\in[K]
\end{align}
And for a $\hat\btheta$ given, an estimate of $\bTau$ obtained by the second optimization problem is
\begin{align}
\label{em-sols-tau}
\hat\tau_{i,k}&= \frac{\hat\pi_k\varphi_{\hat\bmu_k,\hat\bOmega_k}(\bx_i)}{\sum_{k'\in[K]}\hat\pi_{k'}\varphi_{\hat\bmu_{k'},\hat\bOmega_{k'}}(\bx_i)},\qquad\forall k\in[K],\ \forall i\in[n].
\end{align}

However, due to the penality $\lambda_k||\bOmega_k||_{1,1}$, the estimation of $\bOmega_k$ is not straightforward.

We introduce the weighted empirical covariance matrix
\begin{equation}
\bSigma_{n,k} = \frac{1}{n}\frac{\sum_{i=1}^n\tau_{i,k}(\bx_i-\hat\bmu_k)(\bx_i-\hat\bmu_k)^\top}{\sum_{i=1}^n\tau_{i,k}}
\end{equation}
The Gaussian density in equation \eqref{cost_fun_pen} can be expanded as follows


\begin{align*}
\label{cost_fun_pen_2}
F^{pen}(\btheta,\bTau)  =& \sum_{k=1}^K \bigg( \sum_{i=1}^{n}\Big\{ \tau_{i,k} \Big(
-\frac{p}{2}\log(2\pi)+\frac{1}{2}\log|\bOmega_k|\
&-\frac{1}{2}(\bx_i-\bmu_k)^T\bOmega_k(\bx_i-\bmu_k) \Big)+\tau_{i,k} \log(\pi_k/\tau_{i,k})\Big\} -\lambda_k||\bOmega_k||_{1,1}\bigg)\
=& -\frac{np}{2}\log(2\pi)+\sum_{k=1}^K \bigg(\frac{n\pi_k}{2}\log|\bOmega_k|\
&+\sum_{i=1}^{n}\Big\{ -\frac{\tau_{i,k}}{2}(\bx_i-\bmu_k)^T\bOmega_k(\bx_i-\bmu_k)+\tau_{i,k} \log(\pi_k/\tau_{i,k})\Big\} -\lambda_k||\bOmega_k||_{1,1}\bigg) 
\end{align*}

The opposite minimization problem regarding each $\bOmega_k$ is
\begin{equation}
\bOmega_k \in \argmin_{ \bOmega\succeq 0}\Big\{-\frac{n\pi_k}{2}\log|\bOmega|+\frac{1}{2}\sum_{i=1}^{n}\tau_{i,k}(\bx_i-\bmu_k)^T\bOmega(\bx_i-\bmu_k)+\lambda_k||\bOmega||_{1,1}\Big\}
\end{equation}
Using the well-known commutativity property of the trace operator and dividing by $n\pi_k$
\begin{equation}
\bOmega_k \in \argmin_{ \bOmega\succeq 0} \Big\{ -\frac{1}{2}\log|\bOmega| +\frac{1}{2} tr(\bSigma_{n,k}\bOmega)+\frac{\lambda_k}{n\pi_k}||\bOmega||_{1,1}\Big\}
\end{equation}

Our algorithm solves a graphical lasso problem within each cluster. We use a block coordinate ascent algorithm \citep{mazum_lasso} to solve this convex problem as in the graphical lasso implementation in R \url{http://statweb.stanford.edu/~tibs/glasso/} \


The alternating maximization procedure is summarized in the following algorithm.


\begin{figure}[H]
\begin{center}
\mybox{
\begin{minipage}{0.85\linewidth}
\begin{algorithmic}%\SetAlgoLined\tt\SetLine
\small
\STATE {\bfseries Input:} data vectors $\bx_1,\ldots,\bx_n\in\RR^p$ and the number of clusters $K$
\STATE {\bfseries Output:} parameter estimate $\hat\btheta = \{\hat\bmu_k,\hat\bOmega_k,\hat\pi_k\}_{k\in[K]}$
\STATE {\tt 1: Initialize $t=0$, $\btheta=\btheta^0$.}
\STATE {\tt 2: {\bf Repeat}}
\STATE {\tt 3: \qquad Update the parameter $\bTau$:}
\begin{align*}
\tau_{i,k}^{t}  &= \frac{\pi_k^{t}\varphi_{\bmu_k^{t},\bOmega_k^{t}}(\bx_i)}{\sum_{k'\in[K]}\pi^{t}_{k'}\varphi_{\bmu^{t}_{k'},\bOmega^{t}_{k'}}(\bx_i)}.
\end{align*}
\STATE {\tt 4: \qquad Update the parameter $\btheta$:}
\begin{align*}
\pi_k^{t+1}     &= \frac1n\sum_{i=1}^n \tau_{i,k}^t,\qquad \
\bmu_k^{t+1}    &= \frac1{n\pi_k^{t+1}}\sum_{i=1}^n \tau_{i,k}^t\bx_i\
\bSigma_{n,k}         &= \frac{1}{n^2\pi_k^{t+1}}\sum_{i=1}^n\tau_{i,k}^{t+1}(\bx_i-\hat\bmu_k^{t+1})(\bx_i-\hat\bmu_k^{t+1})^\top\
\bOmega_k^{t+1} & \in \argmin_{ \bOmega\succeq 0} \Big\{ -\frac{1}{2}\log| \bOmega |+\frac{1}{2} tr(\bSigma_{N,k}\bOmega)+\frac{\lambda_k}{n\pi_k^{t+1}}||\bOmega||_{1,1}\Big\}
\end{align*}
\STATE {\tt 5: \qquad increment $t$: $t=t+1$}.
\STATE {\tt 6: {\bf Until} stopping rule.}
\STATE {\tt 7: {\bf Return} $\btheta^{t}$}.
\end{algorithmic}
\end{minipage}}
   \caption{Graphical lasso algorithm for Gaussian mixtures}
   \label{algo:PEM}
\end{center}
\end{figure}
