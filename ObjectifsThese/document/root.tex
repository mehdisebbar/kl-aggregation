\documentclass[a4paper,12pt]{article}
\usepackage[utf8]{inputenc}
\usepackage[T1]{fontenc}

\usepackage{amssymb}
\usepackage{amsmath}
\usepackage{amsthm}
\usepackage{amsopn}
\usepackage{graphicx}
\usepackage{mathrsfs}
\usepackage{empheq}

%\usepackage{psfrag}
\usepackage{float}
\usepackage{amsfonts}
%\usepackage{epstopdf}
\usepackage{dsfont}
%\usepackage{xcolor,fancybox}
\usepackage{url}
\usepackage{pifont}

% For algorithms
\usepackage{algorithm}
\usepackage{algorithmic}
\usepackage{caption}
% Need it for \caption*
\usepackage{xspace}
% Fix macro spacing bug

\usepackage{tcolorbox}
\definecolor{mycolor}{rgb}{0.122, 0.435, 0.698}

\newtcbox{\mybox}{nobeforeafter,colframe=mycolor,colback=mycolor!10!white,boxrule=0.5pt,arc=4pt,
  boxsep=0pt,left=6pt,right=6pt,top=6pt,bottom=6pt,tcbox raise base}

\usepackage{tikz}
\usetikzlibrary{fit,positioning}
\usetikzlibrary{arrows}


%\usepackage{showkeys}

\usepackage{natbib}
\bibliographystyle{plainnat}
\bibpunct{(}{)}{;}{a}{,}{,}
\usepackage{hyperref}
\hypersetup{
  colorlinks = true,
  urlcolor = blue,
  linkcolor = blue,
  citecolor = blue,
}

% Packages hyperref and algorithmic misbehave sometimes.  We can fix
% this with the following command.
\newcommand{\theHalgorithm}{\arabic{algorithm}}


\usepackage{geometry}
\geometry{a4paper, left=30mm, right=30mm, top=30mm, bottom=30mm, nohead}




%\setlength{\arraycolsep}{2pt}
%\setlength{\parskip}{.04in}
%\setlength{\footskip}{30pt}

\let\bb\mathbb       % BlackBoardBold (double letters)


  \def\AA{{\bb A}}\def\CC{{\bb C}}\def\DD{{\bb D}}\def\EE{{\bb E}}
  \def\GG{{\bb G}}\def\HH{{\bb H}}\def\KK{{\bb K}}\def\LL{{\bb L}}
  \def\MM{{\bb M}}\def\QQ{{\bb Q}}\def\TT{{\bb T}}\def\YY{{\bb Y}}
  \def\PP{{\bb P}}\def\II{{\bb I}}\def\WW{{\bb W}}\def\XX{{\bb X}}
  \def\VV{{\bb V}}\def\SS{{\bb S}}\def\BB{{\bb B}}\def\NN{{\bb N}}
  \def\RR{{\bb R}}\def\ZZ{{\bb Z}}\def\FF{{\bb F}}\def\DD{{\bb D}}
  \def\OO{{\bb O}}\def\JJ{{\bb J}}\def\UU{{\bb U}}

\def\cH{\mathcal H}
\def\cY{\mathcal Y}
\def\cX{\mathcal X}
\def\cA{\mathcal A}
\def\mC{\mathcal C}
\def\Ex{\mathbf E}

\def\bb{\mathbb}
%end of header
\def\hat{\widehat}
\def\bfX{\mathbf X}
\def\bfB{\mathbf B}
\def\bfA{\mathbf A}
\def\bSigma{\boldsymbol\Sigma}
\def\bOmega{\boldsymbol\Omega}
\def\bmu{\boldsymbol\mu}
\def\bnu{\boldsymbol\nu}
\def\btau{\boldsymbol\tau}
\def\bTau{\boldsymbol{\mathcal T}}
\def\btheta{{\boldsymbol\theta}}
\def\bTheta{{\boldsymbol\Theta}}
\def\bPi{\boldsymbol\Pi}
\def\bX{\boldsymbol X\!}
\def\bx{\boldsymbol x}
\def\bbeta{\boldsymbol \beta}
\def\bY{\boldsymbol Y}
\def\bxi{\boldsymbol \xi}
\def\bpi{\boldsymbol \pi}
\def\ci{\perp\!\!\!\perp}
\def\bZ{\boldsymbol Z}
\def\11{\mathds 1}
\def\b1{\mathbf 1}
\def\Pb{\mathbf P}

\DeclareMathOperator*{\argmax}{arg\,max}
\DeclareMathOperator*{\argmin}{arg\,min}

\parindent=0pt
\parskip=5pt
\renewcommand{\baselinestretch}{1.08}


\newtheorem{rem}{Remark}
\newtheorem{defi}{Definition}
\newtheorem{lem}{Lemma}
\newtheorem{cor}{Corollary}
\newtheorem{pro}{Proposition}
\newtheorem{theo}{Theorem}
\newtheorem{fact}{Fact}


\title{\vspace{-60pt}~\\\textbf{\textsf{Double sparsity in high-dimensional Gaussian mixture estimation and clustering}}\\[20pt]
\textsf{Subject Overview}}
\author{\vspace{-20pt}~\\ \textsf{Supervisor: A.S. Dalalyan}\\
\textsf{PHd Student: M. Sebbar}}
\date{\today}

\begin{document}
\maketitle
\tableofcontents
\newpage
\section{Notation}
For any vector $\bmu\in\RR^p$ and $p\times p$ matrix $\bSigma$ we denote by $\varphi_{\bmu,\bSigma}$ the probability density
function of the Gaussian distribution $\mathcal N_p(\bmu,\bSigma)$ with mean $\bmu$ and covariance matrix $\bSigma$.
The transpose of a matrix $\bfA $ is denoted by $\bfA ^\top$ while $|\bfA |$ stands for its determinant if $\bfA $ is a square matrix.
The set of $p\times p$ positive semidefinite matrices is denoted $\mathcal S_+^p$, and the set of positive definite
matrices is denoted by $\mathcal S_{++}^p$. We write $\b1_p$ for the $p$-vector $(1,\ldots,1)^\top$. For any integer $K>0$, we
define $[K]$ as the set $\{1,\ldots,K\}$.


\section{Introduction}

\input{introduction.tex}

\newpage
%%%%%%%%%%%%%%%%%%%%%%

\section{Algorithm 1: penalized-EM approach}
This algorithm is inspired by the EM algorithm. The penalized log-likelihood of the Gaussian mixture model is

\begin{equation}\label{pen-log-likelihood}
\ell_n^{pen}(\btheta)=\sum_{i=1}^{n}\log{p_{\btheta}(\bx_i)}+pen(\btheta)=
\sum_{i=1}^{n}\log\bigg\{{\sum_{k=1}^K\pi_k\varphi_{(\bmu_{k},\bSigma_{k})}(\bx_i)}\bigg\} +pen(\btheta) .
\end{equation}

Using the same representation as \ref{eq:3} we get:

\begin{align}\label{eq:4}
\hat\btheta
    &\in\argmax_{\btheta=(\bpi,\bmu,\bSigma)}\max_{\bTau}
    \bigg\{\sum_{i=1}^{n} \sum_{k=1}^K \Big\{\tau_{i,k}\log\varphi_{\bmu_{k},\bSigma_{k}}(\bx_i)+\tau_{i,k}
    \log(\pi_k/\tau_{i,k})\Big\}+pen(\btheta)\bigg\}.
\end{align}

In this problem, we suppose that each feature $Y_i$ is independent from $Y_j$ given the other features $Y_l \text{ with } l\in[p]$. This property is evaluated by the sparsity of $\bOmega_k=\bSigma_k^{-1}$ given a cluster $k$.
Therefore, we chose $pen(\theta_k)=\lambda_k||\bOmega_k||_{1,1}$ with $\lambda_k\in\RR^+$.\\

The optimization problem of the cost function
\begin{equation}
\label{cost_fun_pen}
F^{pen}(\btheta,\bTau)  = \sum_{i=1}^{n} \sum_{k=1}^K \Big\{\tau_{i,k}\log\varphi_{\bmu_{k},\bSigma_{k}}(\bx_i)+\tau_{i,k}
    \log(\pi_k/\tau_{i,k})\Big\}-\lambda_k||\bOmega_k||_{1,1}.
\end{equation}
is the same as the cost function defined in lemma \ref{lemma1} regarding the variables $(\bpi,\bmu)$ and $\bTau$ and expressions are given in \ref{em-sols} and \ref{em-sols-tau}. However, the penality is exploited in the optimization regarding $\bSigma_k$.\\

We introduce
\begin{equation}
    S_{N,k}=\frac{\sum_{n=1}^N\tau_{n,k}(x_n-\hat\mu_k)(x_n-\hat\mu_k)^\top}{\sum_{n=1}^N\tau_{n,k}}
\end{equation}
For a given $k\in[K]$, considering that $\bOmega_k=\bSigma_k^{-1}$ and the opposite minimization problem regarding $\bOmega_k$, the  equation \ref{cost_fun_pen} can be rewritten as:
\begin{equation}
    F^{pen}(\btheta,\bTau)  = -\frac{1}{2}\log| \Omega |-\frac{1}{2} tr(S_{N,k}\Omega)+\lambda_k||\Omega||_{1,1}
\end{equation}
and the minimization problem is:
\begin{equation}
    \bOmega_k \in \argmin_{ \Omega\succeq 0} \Big\{ -\frac{1}{2}\log| \Omega |-\frac{1}{2} tr(S_{N,k}\Omega)+\lambda_k||\Omega||_{1,1}\Big\}
\end{equation}


\begin{figure}[H]
\begin{center}
\mybox{
\begin{minipage}{0.85\linewidth}
\begin{algorithmic}%[1]\tt
%\SetLine%\SetAlgoLined
\small
\STATE {\bfseries Input:} data vectors $\bx_1,\ldots,\bx_n\in\RR^p$ and the number of clusters $K$
\STATE {\bfseries Output:} parameter estimate $\hat\btheta = \{\hat\bmu_k,\hat\bOmega_k,\pi_k\}_{k\in[K]}$
\STATE {\tt 1: Initialize $t=0$, $\btheta=\btheta^0$.}
\STATE {\tt 2: {\bf Repeat}}
\STATE \qquad {\tt 3: Update the parameter $\bTau$:}
\begin{align*}
\tau_{i,k}^{t}  &= \frac{\pi_k^{t}\varphi_{\bmu_k^{t},\bOmega_k^{t}}(\bx_i)}{\sum_{k'\in[K]}\pi^{t}_{k'}\varphi_{\bmu^{t}_{k'},\bOmega^{t}_{k'}}(\bx_i)}.
\end{align*}
\STATE \qquad{\tt 4: Update the parameter $\btheta$:}
\begin{align*}
\pi_k^{t+1}     &= \frac1n\sum_{i=1}^n \tau_{i,k}^t,\qquad
\bmu_k^{t+1}    = \frac1{n\pi_k^{t+1}}\sum_{i=1}^n \tau_{i,k}^t\bx_i,\\
\bOmega_k^{t+1} & \in \argmin_{ \Omega\succeq 0} \Big\{ -\frac{1}{2}\log| \Omega |-\frac{1}{2} tr(S_{N,k}\Omega)+\lambda_k||\Omega||_{1,1}\Big\} \\.
\end{align*}
\STATE \qquad {\tt 5: increment $t$: $t=t+1$}.
\STATE {\tt 6: {\bf Until} stopping rule.}
\STATE {\tt 7: {\bf Return} $\btheta^{t}$}.
\end{algorithmic}
\end{minipage}}
   \caption{Penalised EM algorithm for Gaussian mixtures}
   \label{algo:PEM}
\end{center}
\end{figure}

\section{Algorithm 2}
We consider the diagonal matrix $D_{\lambda}=diag(\lambda_1,\dots,\lambda_K)$.
\begin{figure}[H]
\begin{center}
\mybox{
\begin{minipage}{0.85\linewidth}
\begin{algorithmic}%[1]\tt
%\SetLine%\SetAlgoLined
\small
\STATE {\bfseries Input:} data vectors $\bx_1,\ldots,\bx_n\in\RR^p$, the number of clusters $K$ and $D_{\lambda}$
\STATE {\bfseries Output:} parameter estimate $\hat\btheta = \{\hat\bmu_k,\hat\bOmega_k,\pi_k\}_{k\in[K]}$
\STATE {\tt 1: Initialize $t=0$, $\btheta=\btheta^0$.}
\STATE {\tt 2: {\bf Repeat}}
\STATE \qquad {\tt 3: Update the parameter $\bTau$:}
\begin{align*}
\tau_{i,k}^{t}  &= \frac{\pi_k^{t}\varphi_{\bmu_k^{t},\bOmega_k^{t}}(\bx_i)}{\sum_{k'\in[K]}\pi^{t}_{k'}\varphi_{\bmu^{t}_{k'},\bOmega^{t}_{k'}}(\bx_i)}.
\end{align*}
\STATE \qquad{\tt 4: Update the parameter $\btheta$:}
\begin{align*}
(\mu^k,B^k)&\in \argmin_{(\mu,B)\in \RR^p \times \RR^{p\times p},B_{jj}=1}\Big\{\frac{1}{N}\sum_{n=1}^N\tau_n^k(t)||(x_n-\mu)^TB||^2_2+||D_{\lambda}B||_{1,1}\Big\}\\
\pi_k^{t+1}     &= \\
\bmu_k^{t+1}    &= \\
\bOmega_k^{t+1} &
\end{align*}
\STATE \qquad {\tt 5: increment $t$: $t=t+1$}.
\STATE {\tt 6: {\bf Until} stopping rule.}
\STATE {\tt 7: {\bf Return} $\btheta^{t}$}.
\end{algorithmic}
\end{minipage}}
   \caption{Lasso for Gaussian mixtures}
   \label{algo:LassoGM}
\end{center}
\end{figure}


\input{lasso.tex}



\section{Comments}
\bibliographystyle{plain}
\bibliography{ref}

\end{document}
